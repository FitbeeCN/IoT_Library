\documentclass{article}

\usepackage[latin1]{inputenc}
\usepackage{tikz}
\usetikzlibrary{shapes,arrows,positioning,calc}
\begin{document}
\pagestyle{empty}

% Define block styles
\tikzstyle{decision} = [diamond, aspect=1.8, draw, fill=blue!10,
    text width=12em,
    minimum width=20em,
    text badly centered, inner sep=0pt]
\tikzstyle{start} = [rectangle, draw, fill=green!10, 
    text width=12em, text centered, rounded corners=12, minimum height=4em]
\tikzstyle{stop} = [rectangle, draw, fill=green!10, 
    text width=12em, text centered, rounded corners=12, minimum height=4em]
\tikzstyle{block} = [rectangle, draw, fill=blue!10, 
    text width=12em, text centered, minimum height=4em]
\tikzstyle{line} = [draw, -latex']
\tikzstyle{cloud} = [draw, ellipse,fill=red!20, node distance=3cm,
    minimum height=2em]
    
\begin{figure}
\centering
\begin{tikzpicture}[scale=0.8, transform shape, node distance = 2em, auto]
    % Place nodes
    \node [start] (start) {\tt{\mbox{openLine(hashName)}}};
    \node [decision, below=of start] (islineopen) {Is there an existing line open to this node?};
    \node [decision, below=of islineopen] (haspath) {Is the provided node a seed?  (i.e. we have full path and public key information.)};
    \node [block, right=of haspath] (openline) {Open a line to the node directly.};
    \node [block, below=of openline] (waitforremote2) {Wait for the remote node's {\tt open} response.};
    \node [decision, below=of haspath] (isintable) {Is this hash name in the routing table?};
    \node [block, right=of isintable] (nodelookup) {Perform node lookup to find the node.};
    \node [block, below=of isintable] (peerconnect) {UDP hole-punch if needed, and ask the referring node to introduce us via \tt{peer/connect}.};
    \node [block, below=of peerconnect] (waitforremote1) {Wait for the remote node's {\tt open} packet.};
    \node [block, below=of waitforremote1] (sendopenresponse) {Send our {\tt open} packet in response.};
    \node [stop, below=of sendopenresponse] (lineisopen) {Line is open.};
    % Draw edges
    \path [line] (start) -- (islineopen);
    \path [line] (islineopen) -- node [near start] {no} (haspath);
    \path [line] (haspath) -- node [near start] {yes} (openline);
    \path [line] (haspath) -- node [near start] {no} (isintable);
    \path [line] (openline) -- (waitforremote2);
    \path [line] (isintable) -- node [near start] {no} (nodelookup);
    \path [line] (isintable) -- node [near start] {yes} (peerconnect);
    \path [line] (nodelookup) |- (peerconnect);
    \path [line] (peerconnect) -- (waitforremote1);
    \path [line] (waitforremote1) -- (sendopenresponse);
    \path [line] (sendopenresponse) -- (lineisopen);
    \path [line] (waitforremote2.east) -- ++(0.5,0) |- (lineisopen.east);
    \path [line] (islineopen.west) -- node [near start] {yes} ++(-1.2,0) |- (lineisopen.west);
\end{tikzpicture}
\caption{Steps to open a Telehash line.}
\end{figure}

\end{document}

